\documentclass[../main.tex]{subfiles}

% =============================================
% 【自動設定】単体ビルド時の章番号補正
% ファイル名 (chXX) の数字を読み取り, 自動でセクション番号を調整します
% main.tex からビルドする時は無視されます
% =============================================
\directlua{
  local s, e, num = string.find(tex.jobname, "ch(" .. string.char(37) .. "d+)") %chktex 36 %chktex 18 %chktex 12 %chktex 26
  if num then
    tex.print(string.char(92) .. "setcounter{section}{" .. (tonumber(num) - 1) .. "}")%chktex 36 %chktex 26 %chktex 18 %chktex 12 %chktex 8
  end
}

\begin{document}

\section{常微分方程式}

\subsection{変数分離型}
変数分離型とは, 未知関数の微分方程式が
\[
  \dfrac{\:dy\:}{\:dx\:} = f(x) g(y)
\]
の形に書けるものである. $f(x)$ は $x$ のみ, $g(y)$ は $y$ のみの関数である.
このとき $g(y)\neq 0$ を仮定して両辺を整理して
\[
  \dfrac{\:1\:}{\:g(y)\:}dy = f(x)dx
\]
とし, 両辺を不定積分することで解を求める. すなわち
\[
  \int \dfrac{\:1\:}{\:g(y)\:}dy = \int f(x)dx + \mathrm{const}
\]
となる. ここから $y$ を $x$ の式として整理したものが一般解であり, 初期条件を満たすように積分定数を定めたものが特殊解である. さらに, 一般解の族に含まれないが方程式を満たす解が存在する場合, それを特異解と呼び区別する. また, $g(y)=0$ を満たす定数解が存在する場合, それらも解に含まれる. 

\begin{exproblem}{変数分離型の例題}{ex:separable-exercises}
  次の微分方程式を解け.
  \begin{enumerate}[label=(\arabic*),leftmargin=1.4em] %chktex 36
    \begin{mathpad}
    \item ${(1+x)}\dfrac{\:dy\:}{\:dx\:}=1+y,\quad y(0)=0$
    \end{mathpad}\begin{mathpad}
    \item $y^{3}+x^{6}\dfrac{\:dy\:}{\:dx\:}=0$
    \end{mathpad}\begin{mathpad}
    \item $\dfrac{\:dy\:}{\:dx\:}=\dfrac{\:1\:}{\:\log(3+y+2)+1\:}-3$
    \end{mathpad}\begin{mathpad}
    \item ${(1+x^{2})}\dfrac{\:dy\:}{\:dx\:}+{(1+y^{2})}=0,\quad x=1\ \text{のとき}\ y=\sqrt{3}$
    \end{mathpad}
  \end{enumerate}
\par\noindent\dotfill\par
\noindent\textbf{【解答】}\par
  \begin{pstep}{1}
    $1+x\neq 0$ で
    \begin{mathpad}
      \[
        \dfrac{\:1\:}{\:1+y\:}\hspace{0.1em}\:dy\: = \dfrac{\:1\:}{\:1+x\:}\hspace{0.1em}\:dx\:
      \]
    \end{mathpad}
    と分離できる. 積分して
    \begin{mathpad}
      \[
        \ln \abs{1+y} = \ln \abs{1+x} + C
      \]
    \end{mathpad}\begin{mathpad}
      \[
        1+y = C(1+x)
      \]
    \end{mathpad}
    を得る. 初期条件 $y(0)=0$ から $C=1$ となり
    \begin{mathpad}
      \[
        y = x
      \]
    \end{mathpad}
    である. なお $g(y)=1+y$ とおくと $g(y)=0$ を満たす $y=-1$ も定数解である.
  \end{pstep}
  \begin{pstep}{2}
    $x\neq 0$ で
    \begin{mathpad}
      \[
        \dfrac{\:1\:}{\:y^{3}\:}\hspace{0.1em}\:dy\:=-\dfrac{\:1\:}{\:x^{6}\:}\hspace{0.1em}\:dx\:
      \]
    \end{mathpad}\begin{mathpad}
      \[
        -\dfrac{\:1\:}{\:2^{2}\:}=\dfrac{\:1\:}{\:5^{5}\:}+C
      \]
    \end{mathpad}
    となる. したがって
    \begin{mathpad}
      \[
        \dfrac{\:1\:}{\:y^{2}\:}=-\dfrac{\:2\:}{\:5^{5}\:}+C
      \]
    \end{mathpad}
    が一般解である. また $y=0$ は定数解である.
  \end{pstep}
  \begin{pstep}{3}
    $u=3+y+2$ とおくと $\dfrac{\:du\:}{\:dx\:}=3+\dfrac{\:dy\:}{\:dx\:}$ であるから
    \begin{mathpad}
      \[
        \dfrac{\:du\:}{\:dx\:}=\dfrac{\:1\:}{\:\log u+1\:}
      \]
    \end{mathpad}\begin{mathpad}
      \[
        (\log u+1)\hspace{0.1em}\:du\:=\:dx\:
      \]
    \end{mathpad}
    を積分して
    \begin{mathpad}
      \[
        u\log u=x+C
      \]
    \end{mathpad}\begin{mathpad}
      \[
        (3+y+2)\log(3+y+2)=x+C
      \]
    \end{mathpad}
    が一般解である.
  \end{pstep}
  \begin{pstep}{4}
    \begin{mathpad}
      \[
        \dfrac{\:1\:}{\:1+y^{2}\:}\hspace{0.1em}\:dy\:=-\dfrac{\:1\:}{\:1+x^{2}\:}\hspace{0.1em}\:dx\:
      \]
    \end{mathpad}\begin{mathpad}
      \[
        \arctan y=-\arctan x+C
      \]
    \end{mathpad}
    となる. 初期条件 $x=1,\ y=\sqrt{3}$ より
    \begin{mathpad}
      \[
        C=\dfrac{\:7\pi\:}{\:12\:}
      \]
    \end{mathpad}\begin{mathpad}
      \[
        \arctan y+\arctan x=\dfrac{\:7\pi\:}{\:12\:}
      \]
    \end{mathpad}
    が特殊解である.
  \end{pstep}
\end{exproblem}

\subsection{一階線形微分方程式の解法}
一階線形微分方程式
\[
  \dfrac{\:dy\:}{\:dx\:} + p(x) y = q(x)
\]
を考える. $q(x)=0$ のとき同次方程式, $q(x)\neq 0$ のとき非同次方程式である.

\begin{sidebox}[TBblue]{一階線形微分方程式の解法}
  \begin{pstep}{解法1}
    ここでは特殊解 $\alpha(x)$ が既に分かっている場合を扱う. $y$ を任意の解, $\alpha(x)$ を特殊解として
    \begin{mathpad}
      \[
        \dfrac{\:dy\:}{\:dx\:} + p(x) y = q(x)
      \]
    \end{mathpad}
    \begin{mathpad}
      \[
        \dfrac{\:d\alpha\:}{\:dx\:} + p(x)\alpha = q(x)
      \]
    \end{mathpad}
    を引くと
    \begin{mathpad}
      \[
        \dfrac{\:d(y-\alpha)\:}{\:dx\:} + p(x)(y-\alpha)=0
      \]
    \end{mathpad}
    となる. $Y=y-\alpha$ とおくと
    \begin{mathpad}
      \[
        \dfrac{\:dY\:}{\:dx\:} + p(x)m=0
      \]
    \end{mathpad}
    であり, 変数分離によって
    \begin{mathpad}
      \[
        Y=C e^{-\int p(x)dx}
      \]
    \end{mathpad}
    を得る. したがって一般解は
    \begin{mathpad}
      \[
        y=\mathnestunderline[blue]{非同次方程式の一般解}{%
          \mathunderline[solid][blue]{C e^{-\int p(x)\dd x}}[同次方程式の一般解]
          +\mathunderline[solid][blue]{\alpha(x)}[特殊解]
        }
      \]
    \end{mathpad}
    となる. この形は, 同次方程式の一般解に特殊解を加えるという標準的な構造を表している.
  \end{pstep}
  \tcbbreak{}
  \begin{pstep}{解法2}
    \begin{pstep}[box]{Step1}
      同次方程式を解いて
      \begin{mathpad}
        \[
        \dfrac{\:dy\:}{\:dx\:}+p(x)y=0
        \]
      \end{mathpad}
      \begin{mathpad}
        \[
          y=C e^{-\int p(x)\dd x}
        \]
      \end{mathpad}
      を得る.
    \end{pstep}
    \begin{pstep}[box]{Step2}
      定数 $C$ を関数 $C(x)$ に変えて非同次方程式に代入する.
      \begin{mathpad}
        \[
          y=C(x)e^{-\int p(x)\dd x}
        \]
      \end{mathpad}
      \begin{mathpad}
        \[
          \dfrac{\:d\:}{\:dx\:}\!\left(C(x)e^{-\int p(x)\dd x}\right)+p(x)C(x)e^{-\int p(x)\dd x}=q(x)
        \]
      \end{mathpad}
      ここで
      \begin{mathpad}
        \[
          C'(x)e^{-\int p(x)\dd x}-p(x)C(x)e^{-\int p(x)\dd x}+p(x)C(x)e^{-\int p(x)\dd x}=q(x)
        \]
      \end{mathpad}
      \begin{mathpad}
        \[
          C'(x)e^{-\int p(x)\dd x}=q(x)
        \]
      \end{mathpad}
      \begin{mathpad}
        \[
          C'(x)=\mathunderline[wave][red]{q(x)e^{\int p(x)\dd x}}[\ensuremath{x}だけの関数]
        \]
      \end{mathpad}
      となる. よって
      \begin{mathpad}
        \[
          C(x)=\int q(x)e^{\int p(x)\dd x}\dd x + C
        \]
      \end{mathpad}
      \begin{mathpad}
        \[
          y=e^{-\int p(x)\dd x}\Biggl\{\int q(x)e^{\int p(x)\dd x}\dd x + C \Biggr\}
        \]
      \end{mathpad}
      を得る. これが非同次方程式の一般解である.
    \end{pstep}
  \end{pstep}

  \begin{pstep}{解法3}
    積分因子を用いる. 方程式
    \begin{mathpad}
      \[
        \dfrac{\:dy\:}{\:dx\:}+p(x)y=q(x)
      \]
    \end{mathpad}
    の両辺に
    \begin{mathpad}
      \[
        \mathunderline[wave][red]{e^{\int p(x)\dd x}}[積分因子]
      \]
    \end{mathpad}
    をかけると
    \begin{mathpad}
      \[
        e^{\int p(x)\dd x}\dfrac{\:dy\:}{\:dx\:}+e^{\int p(x)\dd x}p(x)y
        =e^{\int p(x)\dd x}q(x)
      \]
    \end{mathpad}
    となる. 左辺は積の微分で
    \begin{mathpad}
      \[
        \dfrac{\:d\:}{\:dx\:}\!\left(e^{\int p(x)\dd x}y\right)
        =e^{\int p(x)\dd x}q(x)
      \]
    \end{mathpad}
    であるから, 両辺を積分して
    \begin{mathpad}
      \[
        e^{\int p(x)\dd x}y=\int q(x)e^{\int p(x)\dd x}\dd x + C
      \]
    \end{mathpad}
    を得る. よって一般解は
    \begin{mathpad}
      \[
        y=e^{-\int p(x)\dd x}\Biggl\{\int q(x)e^{\int p(x)\dd x}\dd x + C \Biggr\}
      \]
    \end{mathpad}
    となる.
  \end{pstep}
\end{sidebox}

\begin{exproblem}{一階線形微分方程式の例題}{ex:first-linear-methods}
  次の微分方程式を解け(一般解のみ). (1) は解法1, (2) は解法2, (3), (4) は解法3に対応する.
  \begin{enumerate}[label=(\arabic*),leftmargin=1.4em] %chktex 36
    \begin{mathpad}
    \item $\dfrac{\:dy\:}{\:dx\:}+2y=e^{-x}$
    \end{mathpad}\begin{mathpad}
    \item $\dfrac{\:dy\:}{\:dx\:}+\dfrac{\:2\:}{\:x\:}y=x^{2}\quad (x>0)$
    \end{mathpad}\begin{mathpad}
    \item $\dfrac{\:dy\:}{\:dx\:}+(1+x)y=x+1$
    \end{mathpad}\begin{mathpad}
    \item $\dfrac{\:dy\:}{\:dx\:}+y=e^{x}$
    \end{mathpad}
  \end{enumerate}
\par\noindent\dotfill\par
\noindent\textbf{【解答】}\par
  \begin{pstep}{1}
    同次解 $y=C e^{-2x}$ を用いて
    \begin{mathpad}
      \[
        y=C e^{-2x}+\alpha(x),\quad \dfrac{\:d\alpha\:}{\:dx\:}+2\alpha=e^{-x}
      \]
    \end{mathpad}
    を満たす特殊解を求めると
    \begin{mathpad}
      \[
        \alpha(x)=-e^{-x}
      \]
    \end{mathpad}
    であるから一般解は
    \begin{mathpad}
      \[
        y=C e^{-2x}-e^{-x}
      \]
    \end{mathpad}
    となる.
  \end{pstep}
  \begin{pstep}{2}
    同次解 $y=C x^{-2}$ とおき, 定数を $C(x)$ に変える.
    \begin{mathpad}
      \[
        y=C(x)x^{-2}
      \]
    \end{mathpad}
    であり
    \begin{mathpad}
      \[
        C'(x)x^{-2}=x^{2}
      \]
    \end{mathpad}
    となる. よって
    \begin{mathpad}
      \[
        C(x)=\dfrac{\:1\:}{\:5\:}x^{5}+C
      \]
    \end{mathpad}
    であるから一般解は
    \begin{mathpad}
      \[
        y=\dfrac{\:1\:}{\:5\:}x^{3}+C x^{-2}
      \]
    \end{mathpad}
    となる.
  \end{pstep}
  \begin{pstep}{3}
    積分因子は
    \begin{mathpad}
      \[
        e^{\int (1+x)\dd x}=e^{x+\dfrac{\:1\:}{\:2\:}x^{2}}
      \]
    \end{mathpad}
    であるから
    \begin{mathpad}
      \[
        \dfrac{\:d\:}{\:dx\:}\!\left(e^{x+\dfrac{\:1\:}{\:2\:}x^{2}}y\right)=(x+1)e^{x+\dfrac{\:1\:}{\:2\:}x^{2}}
      \]
    \end{mathpad}
    となる. ここで
    \begin{mathpad}
      \[
        t=x+\dfrac{\:1\:}{\:2\:}x^{2}\quad \Rightarrow \quad \dfrac{\:dt\:}{\:dx\:}=x+1
      \]
      \[
        (x+1)e^{x+\dfrac{\:1\:}{\:2\:}x^{2}}\dd x=e^{t}\dd t
      \]
    \end{mathpad}
    であるから, 両辺を積分して
    \begin{mathpad}
      \[
        e^{x+\dfrac{\:1\:}{\:2\:}x^{2}}y=e^{x+\dfrac{\:1\:}{\:2\:}x^{2}}+C
      \]
    \end{mathpad}
    を得る. よって一般解は
    \begin{mathpad}
      \[
        y=1+Ce^{-x-\dfrac{\:1\:}{\:2\:}x^{2}}
      \]
    \end{mathpad}
    となる.
  \end{pstep}
  \begin{pstep}{4}
    (3) と同様に積分因子 $e^{x}$ を用いる.
    \begin{mathpad}
      \[
        \dfrac{\:d\:}{\:dx\:}\!\left(e^{x}y\right)=e^{2x}
      \]
    \end{mathpad}
    であるから
    \begin{mathpad}
      \[
        e^{x}y=\dfrac{\:1\:}{\:2\:}e^{2x}+C
      \]
    \end{mathpad}
    を得る. したがって一般解は
    \begin{mathpad}
      \[
        y=\dfrac{\:1\:}{\:2\:}e^{x}+Ce^{-x}
      \]
    \end{mathpad}
    となる.
  \end{pstep}
\end{exproblem}

\subsection{二階線形微分方程式}
\noindent\textbf{二階線形微分方程式}\par
  \begin{mathpad}
    \[
      y''+a(x)y'+b(x)y=f(x)
    \]
  \end{mathpad}
の一般解を考える. 

\begin{sidebox}[TBblue]{二階線形微分方程式の解法}


  \begin{pstep}{解法1}
    特殊解 $\alpha(x)$ が既知の場合を考える. まず同次方程式 $y''+a(x)y'+b(x)y=0$ を解く. 係数が定数なら特性方程式を解いて $y_{1},y_{2}$ を求める. $y$ を任意の解とすると
    \begin{mathpad}
      \[
        y''+a(x)y'+b(x)y=f(x)
      \]
      \[
        \alpha''+a(x)\alpha'+b(x)\alpha=f(x)
      \]
    \end{mathpad}
    であるから, 差を取って
    \begin{mathpad}
      \[
        (y-\alpha)''+a(x)(y-\alpha)'+b(x)(y-\alpha)=0
      \]
    \end{mathpad}
    を得る. $Y=y-\alpha$ とおくと
    \begin{mathpad}
      \[
        Y''+a(x)Y'+b(x)Y=0\quad (\text{同次方程式})
      \]
    \end{mathpad}
    であり, 一次独立な解 $y_{1},y_{2}$ を用いて
    \begin{mathpad}
      \[
        Y=C_{1}y_{1}+C_{2}y_{2}
      \]
    \end{mathpad}
    となる. よって
    \begin{mathpad}
      \[
        y=C_{1}y_{1}+C_{2}y_{2}+\alpha(x)
      \]
    \end{mathpad}
    が一般解である. 特殊解を一般解に加える形になる.
  \end{pstep}

  \begin{pstep}{解法2}
    \begin{pstep}[box]{Step1}
      まず同次方程式
      \begin{mathpad}
        \[
          y''+a(x)y'+b(x)y=0
        \]
      \end{mathpad}
      を解き, 一次独立な解 $y_{1},y_{2}$ を得たとする. 係数が定数なら特性方程式を解いて $y_{1},y_{2}$ を求める. ここで
      \begin{mathpad}
        \[
          L[y]=y''+a(x)y'+b(x)y
        \]
      \end{mathpad}
      とおけば $L$ は微分と加法・定数倍からできているため
      \begin{mathpad}
        \[
          \begin{aligned}
            L[C_{1}y_{1}+C_{2}y_{2}]
            &=(C_{1}y_{1}+C_{2}y_{2})''+a(x)(C_{1}y_{1}+C_{2}y_{2})'+b(x)(C_{1}y_{1}+C_{2}y_{2}) \\
            &=C_{1}(y_{1}''+a(x)y_{1}'+b(x)y_{1})+C_{2}(y_{2}''+a(x)y_{2}'+b(x)y_{2})
          \end{aligned}
        \]
      \end{mathpad}
      となる. よって $L[y_{1}]=L[y_{2}]=0$ から
      \[
        L[C_{1}y_{1}+C_{2}y_{2}]=0
      \]
      が成り立ち, どんな定数 $C_{1},C_{2}$ に対しても
      \begin{mathpad}
        \[
          y=C_{1}y_{1}+C_{2}y_{2}
        \]
      \end{mathpad}
      は同次方程式の解である. 逆に, 初期値
      \begin{mathpad}
        \[
          y(x_{0})=A,\quad y'(x_{0})=B
        \]
      \end{mathpad}
      を満たす解を作りたいとき, $x=x_{0}$ を代入すると
      \begin{mathpad}
        \[
          \begin{pmatrix}
            y_{1}(x_{0}) & y_{2}(x_{0}) \\
            y_{1}'(x_{0}) & y_{2}'(x_{0})
          \end{pmatrix}
          \begin{pmatrix}
            C_{1} \\
            C_{2}
          \end{pmatrix}
          =
          \begin{pmatrix}
            A \\
            B
          \end{pmatrix}
        \]
      \end{mathpad}
      となる. ここで行列の行列式
      \begin{mathpad}
        \[
          W(y_{1},y_{2})=
          \begin{vmatrix}
            y_{1} & y_{2} \\
            y_{1}' & y_{2}'
          \end{vmatrix}
          \neq 0
        \]
      \end{mathpad}
      が必要であり, そのとき任意の初期値 $(A,B)$ に対して一意の $C_{1},C_{2}$ が定まる. よって一般解は $y=C_{1}y_{1}+C_{2}y_{2}$ と書ける.
    \end{pstep}
    \begin{pstep}[box]{Step2}
      $C_{1},C_{2}$ を関数に変えて
      \begin{mathpad}
        \[
          y=C_{1}(x)y_{1}+C_{2}(x)y_{2}
        \]
      \end{mathpad}
      とおき, 付随条件
      \begin{mathpad}
        \[
          C_{1}'(x)y_{1}+C_{2}'(x)y_{2}=0
        \]
      \end{mathpad}
      を課す. このとき
      \begin{mathpad}
        \[
          y'=C_{1}y_{1}'+C_{2}y_{2}'
        \]
        \end{mathpad}\begin{mathpad}
        \[
          y''=C_{1}y_{1}''+C_{2}y_{2}''+C_{1}'y_{1}'+C_{2}'y_{2}'
        \]
      \end{mathpad}
      となるので, 元の方程式に代入して
      \begin{mathpad}
        \[
          C_{1}'(x)y_{1}'+C_{2}'(x)y_{2}'=f(x)
        \]
      \end{mathpad}
      を得る. よって
      \begin{mathpad}
        \[
          \begin{cases}
            C_{1}'(x)y_{1}+C_{2}'(x)y_{2}=0 \\
            C_{1}'(x)y_{1}'+C_{2}'(x)y_{2}'=f(x)
          \end{cases}
        \]
      \end{mathpad}
      である. すなわち
      \begin{mathpad}
        \[
          \begin{pmatrix}
            y_{1} & y_{2} \\
            y_{1}' & y_{2}'
          \end{pmatrix}
          \begin{pmatrix}
            C_{1}'(x) \\
            C_{2}'(x)
          \end{pmatrix}
          =
          \begin{pmatrix}
            0 \\
            f(x)
          \end{pmatrix}
        \]
      \end{mathpad}
      である. ロンスキアン
      \begin{mathpad}
        \[
          W(y_{1},y_{2})=
          \begin{vmatrix}
            y_{1} & y_{2} \\
            y_{1}' & y_{2}'
          \end{vmatrix}
        \]
      \end{mathpad}
      を用いると
      \begin{mathpad}
        \[
          C_{1}'(x)=-\dfrac{\:y_{2}f(x)\:}{\:W(y_{1},y_{2})\:},\quad
          C_{2}'(x)=\dfrac{\:y_{1}f(x)\:}{\:W(y_{1},y_{2})\:}
        \]
      \end{mathpad}
      となる. したがって
      \begin{mathpad}
        \[
          C_{1}(x)=-\int \dfrac{\:y_{2}f(x)\:}{\:W(y_{1},y_{2})\:}\dd x,\quad
          C_{2}(x)=\int \dfrac{\:y_{1}f(x)\:}{\:W(y_{1},y_{2})\:}\dd x
        \]
      \end{mathpad}
      であり, 特殊解は
      \begin{mathpad}
        \[
          \alpha(x)=-y_{1}\int \dfrac{\:y_{2}f(x)\:}{\:W(y_{1},y_{2})\:}\dd x
          +y_{2}\int \dfrac{\:y_{1}f(x)\:}{\:W(y_{1},y_{2})\:}\dd x
        \]
      \end{mathpad}
      となる.
    \end{pstep}
  \end{pstep}
\end{sidebox}

\begin{sidebox}[TBgray]{特性方程式の妥当性}
  係数が定数の同次方程式 $y''+ay'+by=0$ では, $y=e^{rx}$ と仮定すると
  \[
    (r^{2}+ar+b)e^{rx}=0
  \]
  となる. $e^{rx}\neq 0$ であるから $r^{2}+ar+b=0$ を解けばよい. 実根が 2 つなら $e^{r_{1}x},e^{r_{2}x}$, 重解なら $e^{rx},x e^{rx}$ が解となり, これらの線形結合が同次方程式の解を与える.
\end{sidebox}

\begin{exproblem}{二階線形微分方程式の例題}{ex:second-linear-vop}
  次の微分方程式を解け(一般解のみ).
  \begin{enumerate}[label=(\arabic*),leftmargin=1.4em] %chktex 36
    \begin{mathpad}
    \item $y''-y=e^{x}$
    \end{mathpad}\begin{mathpad}
    \item $y''+2y'+y=x$
    \end{mathpad}
    \begin{mathpad}
    \item $y''-3y'+2y=2x^{2}$
    \end{mathpad}
  \end{enumerate}
\par\noindent\dotfill\par
\noindent\textbf{【解答】}\par
  \begin{pstep}{1}
    同次方程式 $y''-y=0$ は係数が定数なので特性方程式
    \[
      r^{2}-1=0
    \]
    を解く. よって
    \begin{mathpad}
      \[
        y_{1}=e^{x},\quad y_{2}=e^{-x}
      \]
    \end{mathpad}
    である. よって
    \begin{mathpad}
      \[
        W(y_{1},y_{2})=
        \begin{vmatrix}
          e^{x} & e^{-x} \\
          e^{x} & -e^{-x}
        \end{vmatrix}
        =-2
      \]
    \end{mathpad}
    となる. 定数変化法より
    \begin{mathpad}
      \[
        C_{1}'(x)=-\dfrac{y_{2}f(x)}{W(y_{1},y_{2})}
        =-\dfrac{e^{-x}\cdot e^{x}}{-2}=\dfrac{\:1\:}{\:2\:}
      \]
      \end{mathpad}\begin{mathpad}
      \[
        C_{2}'(x)=\dfrac{y_{1}f(x)}{W(y_{1},y_{2})}
        =\dfrac{e^{x}\cdot e^{x}}{-2}=-\dfrac{\:1\:}{\:2\:}e^{2x}
      \]
    \end{mathpad}
    であるから
    \begin{mathpad}
      \[
        C_{1}(x)=\dfrac{\:1\:}{\:2\:}x,\quad
        C_{2}(x)=-\dfrac{\:1\:}{\:4\:}e^{2x}
      \]
    \end{mathpad}
    を得る. 特殊解は
    \begin{mathpad}
      \[
        \alpha(x)=C_{1}(x)y_{1}+C_{2}(x)y_{2}
        =\dfrac{\:1\:}{\:2\:}x e^{x}-\dfrac{\:1\:}{\:4\:}e^{x}
      \]
    \end{mathpad}
    となる. よって一般解は
    \begin{mathpad}
      \[
        y=C_{1}e^{x}+C_{2}e^{-x}+\dfrac{\:1\:}{\:2\:}x e^{x}-\dfrac{\:1\:}{\:4\:}e^{x}
      \]
    \end{mathpad}
    である. 定数項は同次解に吸収できるので
    \begin{mathpad}
      \[
        y=C_{1}e^{x}+C_{2}e^{-x}+\dfrac{\:1\:}{\:2\:}x e^{x}
      \]
    \end{mathpad}
    としてよい. ここで $C_{1},C_{2}$ は同次方程式の任意定数である.
  \end{pstep}
  \begin{pstep}{2}
    同次方程式 $y''+2y'+y=0$ は特性方程式
    \[
      r^{2}+2r+1=0
    \]
    を解くと重解 $r=-1$ を持つので
    \begin{mathpad}
      \[
        y_{1}=e^{-x},\quad y_{2}=x e^{-x}
      \]
    \end{mathpad}
    である. よって
    \begin{mathpad}
      \[
        W(y_{1},y_{2})=
        \begin{vmatrix}
          e^{-x} & x e^{-x} \\
          -e^{-x} & (1-x)e^{-x}
        \end{vmatrix}
        =e^{-2x}
      \]
    \end{mathpad}
    となる. 定数変化法より
    \begin{mathpad}
      \[
        C_{1}'(x)=-\dfrac{y_{2}f(x)}{W(y_{1},y_{2})}
        =-\dfrac{x e^{-x}\cdot x}{e^{-2x}}=-x^{2}e^{x}
      \]
      \end{mathpad}\begin{mathpad}
      \[
        C_{2}'(x)=\dfrac{y_{1}f(x)}{W(y_{1},y_{2})}
        =\dfrac{e^{-x}\cdot x}{e^{-2x}}=x e^{x}
      \]
    \end{mathpad}
    である. 積分して
    \begin{mathpad}
      \[
        C_{1}(x)=\int -x^{2}e^{x}\,\dd x=-(x^{2}-2x+2)e^{x}
      \]
      \end{mathpad}\begin{mathpad}
      \[
        C_{2}(x)=\int x e^{x}\,\dd x=(x-1)e^{x}
      \]
    \end{mathpad}
    を得る. 特殊解は
    \begin{mathpad}
      \[
        \alpha(x)=C_{1}(x)y_{1}+C_{2}(x)y_{2}
        =x-2
      \]
    \end{mathpad}
    となる. よって一般解は
    \begin{mathpad}
      \[
        y=C_{1}e^{-x}+C_{2}x e^{-x}+x-2
      \]
    \end{mathpad}
    である. ここで $C_{1},C_{2}$ は同次方程式の任意定数である.
  \end{pstep}
  \begin{pstep}{3}
    同次方程式 $y''-3y'+2y=0$ は特性方程式
    \[
      r^{2}-3r+2=0
    \]
    を解いて
    \begin{mathpad}
      \[
        y_{1}=e^{x},\quad y_{2}=e^{2x}
      \]
    \end{mathpad}
    である. よって
    \begin{mathpad}
      \[
        W(y_{1},y_{2})=
        \begin{vmatrix}
          e^{x} & e^{2x} \\
          e^{x} & 2e^{2x}
        \end{vmatrix}
        =e^{3x}
      \]
    \end{mathpad}
    となる. 定数変化法より
    \begin{mathpad}
      \[
        C_{1}'(x)=-\dfrac{y_{2}f(x)}{W(y_{1},y_{2})}
        =-\dfrac{e^{2x}\cdot 2x^{2}}{e^{3x}}=-2x^{2}e^{-x}
      \]
      \end{mathpad}\begin{mathpad}
      \[
        C_{2}'(x)=\dfrac{y_{1}f(x)}{W(y_{1},y_{2})}
        =\dfrac{e^{x}\cdot 2x^{2}}{e^{3x}}=2x^{2}e^{-2x}
      \]
    \end{mathpad}
    である. 積分して
    \begin{mathpad}
      \[
        C_{1}(x)=\int -2x^{2}e^{-x}\,\dd x=2(x^{2}+2x+2)e^{-x}
      \]
      \end{mathpad}\begin{mathpad}
      \[
        C_{2}(x)=\int 2x^{2}e^{-2x}\,\dd x=-\left(x^{2}+x+\dfrac{\:1\:}{\:2\:}\right)e^{-2x}
      \]
    \end{mathpad}
    を得る. 特殊解は
    \begin{mathpad}
      \[
        \alpha(x)=C_{1}(x)y_{1}+C_{2}(x)y_{2}
        =x^{2}+3x+\dfrac{\:3\:}{\:2\:}
      \]
    \end{mathpad}
    となる. よって一般解は
    \begin{mathpad}
      \[
        y=C_{1}e^{x}+C_{2}e^{2x}+x^{2}+3x+\dfrac{\:3\:}{\:2\:}
      \]
    \end{mathpad}
    である. ここで $C_{1},C_{2}$ は同次方程式の任意定数である.
  \end{pstep}
\end{exproblem}

\subsection{連立常微分方程式の初期値問題}
\begin{sidebox}[TBblue]{連立常微分方程式の初期値問題}
  関数 $x(t),y(t)$ の満たす連立常微分方程式
  \[
    \begin{cases}
      \dfrac{\:dx\:}{\:dt\:}=a x(t)+b y(t) \\
      \dfrac{\:dy\:}{\:dt\:}=c x(t)+d y(t)
    \end{cases}
  \]
  と初期条件
  \[
    x(0)=x_{0},\quad y(0)=y_{0}
  \]
  により与えられる一般解を考える. 
  ここで $a,b,c,d$ は定数である. ベクトル
  \[
    \bm{u}(t)=\begin{pmatrix}x(t)\\y(t)\end{pmatrix},\quad
    M=\begin{pmatrix}a&b\\c&d\end{pmatrix},\quad
    \bm{C}=\begin{pmatrix}x_{0}\\y_{0}\end{pmatrix}
  \]
  とおくと
  \[
    \dfrac{\:d\bm{u}\:}{\:dt\:}=M\bm{u},\quad \bm{u}(0)=\bm{C}
  \]
  と書ける. このとき解は行列指数関数で
  \[
    \bm{u}(t)=e^{Mt}\bm{C}
  \]
  と表される.

  行列指数関数の計算は, $M$ が対角化可能なら $M=PDP^{-1}$ として
  \[
    e^{Mt}=Pe^{Dt}P^{-1}
  \]
  で与えられる. ここで $D=\mathrm{diag}(\lambda_{1},\lambda_{2})$ とおけば
  \[
    e^{Dt}=\mathrm{diag}(e^{\lambda_{1}t},e^{\lambda_{2}t})
  \]
  である.
\end{sidebox}

\begin{exproblem}{連立常微分方程式の例題}{ex:ode-system-basic}
  次の初期値問題を解け.
  \begin{enumerate}[label=(\arabic*),leftmargin=1.4em] %chktex 36
    \begin{mathpad}
    \item
      \[
        \begin{cases}
          \dfrac{\:dx\:}{\:dt\:}=3x+4y \\
          \dfrac{\:dy\:}{\:dt\:}=4x+3y
        \end{cases},\quad
        x(0)=1,\ y(0)=0
      \]
    \end{mathpad}
    \begin{mathpad}
    \item
      \[
        \begin{cases}
          \dfrac{\:dx\:}{\:dt\:}=x+2y \\
          \dfrac{\:dy\:}{\:dt\:}=2x+y
        \end{cases},\quad
        x(0)=0,\ y(0)=1
      \]
    \end{mathpad}
  \end{enumerate}
\par\noindent\dotfill\par
\noindent\textbf{【解答】}\par
  \begin{pstep}{1}
    \[
      M=\begin{pmatrix}3&4\\4&3\end{pmatrix}
    \]
    とおくと, 固有値は $\lambda_{1}=7,\ \lambda_{2}=-1$ であり, 対応する固有ベクトルは
    \[
      \bm{v}_{1}=\begin{pmatrix}1\\1\end{pmatrix},\quad
      \bm{v}_{2}=\begin{pmatrix}1\\-1\end{pmatrix}
    \]
    である. よって
    \[
      P=\begin{pmatrix}1&1\\1&-1\end{pmatrix},\quad
      D=\begin{pmatrix}7&0\\0&-1\end{pmatrix},\quad
      \bm{C}=\begin{pmatrix}1\\0\end{pmatrix}
    \]
    より
    \[
      e^{Dt}=\begin{pmatrix}e^{7t}&0\\0&e^{-t}\end{pmatrix}
    \]
    として
    \[
      \bm{u}(t)=Pe^{Dt}P^{-1}\bm{C}
    \]
    を計算すると
    \[
      \bm{u}(t)=\dfrac{\:1\:}{\:2\:}
      \begin{pmatrix}
        e^{7t}+e^{-t} \\
        e^{7t}-e^{-t}
      \end{pmatrix}
    \]
    となる. したがって
    \[
      x(t)=\dfrac{\:1\:}{\:2\:}\left(e^{7t}+e^{-t}\right),\quad
      y(t)=\dfrac{\:1\:}{\:2\:}\left(e^{7t}-e^{-t}\right)
    \]
    が解である.
  \end{pstep}
  \begin{pstep}{2}
    \[
      M=\begin{pmatrix}1&2\\2&1\end{pmatrix}
    \]
    とおくと, 固有値は $\lambda_{1}=3,\ \lambda_{2}=-1$ であり, 対応する固有ベクトルは
    \[
      \bm{v}_{1}=\begin{pmatrix}1\\1\end{pmatrix},\quad
      \bm{v}_{2}=\begin{pmatrix}1\\-1\end{pmatrix}
    \]
    である. よって
    \[
      P=\begin{pmatrix}1&1\\1&-1\end{pmatrix},\quad
      D=\begin{pmatrix}3&0\\0&-1\end{pmatrix},\quad
      \bm{C}=\begin{pmatrix}0\\1\end{pmatrix}
    \]
    より
    \[
      e^{Dt}=\begin{pmatrix}e^{3t}&0\\0&e^{-t}\end{pmatrix}
    \]
    として
    \[
      \bm{u}(t)=Pe^{Dt}P^{-1}\bm{C}
    \]
    を計算すると
    \[
      \bm{u}(t)=\dfrac{\:1\:}{\:2\:}
      \begin{pmatrix}
        e^{3t}-e^{-t} \\
        e^{3t}+e^{-t}
      \end{pmatrix}
    \]
    となる. したがって
    \[
      x(t)=\dfrac{\:1\:}{\:2\:}\left(e^{3t}-e^{-t}\right),\quad
      y(t)=\dfrac{\:1\:}{\:2\:}\left(e^{3t}+e^{-t}\right)
    \]
    が解である.
  \end{pstep}
\end{exproblem}
\end{document}
