\documentclass[../main.tex]{subfiles}

\begin{document}

\section{重積分}

\subsection{重積分の意味と累次積分}
\noindent\textbf{重積分と累次積分}\par
長方形領域
\[
  D=\{(x,y)\mid a\le x\le b,\ c\le y\le d\}
\]
上で連続な関数 $f(x,y)$ に対し, 重積分は次で定義される:
\begin{mathpad}
  \[
    \iint_{D} f(x,y)\, \dd x \dd y.
  \]
\end{mathpad}
この積分は, $xy$ 平面上の底面 $D$ と高さ $f(x,y)$ で張られる立体の体積を表す.
また, 1変数積分の反復として
\begin{mathpad}
  \[
    \iint_{D} f(x,y)\, \dd x \dd y
    = \int_{a}^{b}\!\left(\int_{c}^{d} f(x,y)\, \dd y\right)\dd x
    = \int_{c}^{d}\!\left(\int_{a}^{b} f(x,y)\, \dd x\right)\dd y
  \]
\end{mathpad}
が成り立つ. この形を累次積分という.

\begin{theorem}{フビニの定理}{thm:fubini}
  閉領域 $D$ 上で $f(x,y)$ が連続ならば, 重積分は累次積分として計算できる:
  \begin{mathpad}
    \[
      \iint_{D} f(x,y)\, \dd x \dd y
      = \int_{a}^{b}\!\left(\int_{g_{1}(x)}^{g_{2}(x)} f(x,y)\, \dd y\right)\dd x
      = \int_{c}^{d}\!\left(\int_{h_{1}(y)}^{h_{2}(y)} f(x,y)\, \dd x\right)\dd y.
    \]
  \end{mathpad}
  領域の表し方に応じて積分順序を選ぶのが基本である.
\end{theorem}

\begin{exproblem}{長方形領域での基本計算}{ex:rect-basic}
  \[
    I=\iint_{D} (x+2y)\, \dd x \dd y,\quad D=\{(x,y)\mid 0\le x\le 2,\ 1\le y\le 3\}
  \]
  を求めよ.
\par\noindent\dotfill\par
\noindent\textbf{【解答】}\par
まず $y$ で内側の積分を行う. 領域が長方形なので境界は定数であり,
累次積分の形がそのまま使える
\begin{mathpad}
  \begin{align*}
    I
      &= \int_{0}^{2}\!\left(\int_{1}^{3}(x+2y)\, \dd y\right)\dd x \\
      &= \int_{0}^{2}\!\left(\biggl[xy+y^{2}\biggr]{}_{\raisebox{-1.1ex}{$y=1$}}^{\raisebox{1.2ex}{$y=3$}}\right)\dd x \\
      &= \int_{0}^{2}\!\left((3x+9)-(x+1)\right)\dd x \\
      &= \int_{0}^{2}\!\left(2x+8\right)\dd x \\
      &= \biggl[x^{2}+8x\biggr]{}_{\raisebox{-1.1ex}{$x=0$}}^{\raisebox{1.2ex}{$x=2$}} \\
      &= 20
  \end{align*}
\end{mathpad}
\end{exproblem}

\subsection{累次積分の設定法}
\noindent\textbf{縦に切る領域($x$ で記述)}\par
領域 $D$ が
\begin{mathpad}
  \[
    D=\{(x,y)\mid a\le x\le b,\ g_{1}(x)\le y\le g_{2}(x)\}
  \]
\end{mathpad}
と表せるとき, $y$ を内側に取る累次積分で計算する:
\begin{mathpad}
  \[
    \iint_{D} f(x,y)\, \dd x \dd y
    = \int_{a}^{b}\!\left(\int_{g_{1}(x)}^{g_{2}(x)} f(x,y)\, \dd y\right)\dd x.
  \]
\end{mathpad}

\noindent\textbf{横に切る領域($y$ で記述)}\par
領域 $D$ が
\begin{mathpad}
  \[
    D=\{(x,y)\mid c\le y\le d,\ h_{1}(y)\le x\le h_{2}(y)\}
  \]
\end{mathpad}
と表せるとき, $x$ を内側に取る累次積分で計算する:
\begin{mathpad}
  \[
    \iint_{D} f(x,y)\, \dd x \dd y
    = \int_{c}^{d}\!\left(\int_{h_{1}(y)}^{h_{2}(y)} f(x,y)\, \dd x\right)\dd y.
  \]
\end{mathpad}

\begin{remark}{積分順序の交換}{rm:order-exchange}
  領域を図示して境界式を読み取り, どちらの順序で書くと簡潔になるかを判断する.
  順序交換では, 境界曲線の上下関係や交点を必ず確認することが重要である.
\end{remark}

\begin{exproblem}{積分順序の交換}{ex:order-change}
  \[
    I=\iint_{D} (x+y)\, \dd x \dd y,\quad D=\{(x,y)\mid 0\le x\le 1,\ x^{2}\le y\le x\}
  \]
  を求めよ.
\par\noindent\dotfill\par
\noindent\textbf{【解答】}\par
まず $y=x^{2}$ と $y=x$ に挟まれる領域を図示する.
$y$ を固定すると $x$ の範囲は $y\le x\le \sqrt{y}$ となるので,
積分順序を交換して計算する
\begin{mathpad}
  \begin{align*}
    I
      &= \int_{0}^{1}\!\left(\int_{y}^{\sqrt{y}}(x+y)\, \dd x\right)\dd y \\
      &= \int_{0}^{1}\!\left(\biggl[\dfrac{\:x^{2}\:}{\:\:2\:}+yx\biggr]{}_{\raisebox{-1.1ex}{$x=y$}}^{\raisebox{1.2ex}{$x=\sqrt{y}$}}\right)\dd y \\
      &= \int_{0}^{1}\!\left(\dfrac{\:y\:}{\:\:2\:}+y^{\tfrac{\:3\:}{\:\:2\:}}-\dfrac{\:y^{2}\:}{\:\:2\:}-y^{2}\right)\dd y \\
      &= \int_{0}^{1}\!\left(\dfrac{\:1\:}{\:\:2\:}y+y^{\tfrac{\:3\:}{\:\:2\:}}-\dfrac{\:3\:}{\:\:2\:}y^{2}\right)\dd y \\
      &= \dfrac{\:3\:}{\:\:20\:}
  \end{align*}
\end{mathpad}
境界の上下関係を誤ると範囲が逆転するため, 図による確認が不可欠である
\end{exproblem}

\subsection{3重積分(累次積分)}
\noindent\textbf{3重積分の累次積分}\par
3次元領域 $E\subset\mathbb{R}^{3}$ 上で連続な関数 $f(x,y,z)$ に対し,
3重積分は次で定義される:
\begin{mathpad}
  \[
    \iiint_{E} f(x,y,z)\, \dd x \dd y \dd z.
  \]
\end{mathpad}
$E$ を
\begin{mathpad}
  \[
    E=\{(x,y,z)\mid a\le x\le b,\ g_{1}(x)\le y\le g_{2}(x),\ h_{1}(x,y)\le z\le h_{2}(x,y)\}
  \]
\end{mathpad}
と書けるとき,
\begin{mathpad}
  \[
    \iiint_{E} f(x,y,z)\, \dd x \dd y \dd z
    = \int_{a}^{b}\!\left(\int_{g_{1}(x)}^{g_{2}(x)}\!\left(\int_{h_{1}(x,y)}^{h_{2}(x,y)} f(x,y,z)\, \dd z\right)\dd y\right)\dd x
  \]
\end{mathpad}
として計算できる. 同様に, $x,y,z$ の順序は状況に応じて入れ替えられる.

\begin{sidebox}{計算手順の要点}
  \begin{enumerate}[label=(\arabic*),leftmargin=2.2em] %chktex 36
    \item 領域を図示し, 断面が単純になる方向を探す.
    \item 内側の積分変数の上下限を, 境界曲面から読み取る.
    \item 外側の積分変数の範囲は, 投影領域の区間で決める.
  \end{enumerate}
  3重積分では, まず $xy$ 平面への投影を描き, その上で $z$ の範囲を決めるのが基本である.
\end{sidebox}

\begin{exproblem}{三角錐領域の体積}{ex:tetra-volume}
  \[
    V=\iiint_{E} 1\, \dd x \dd y \dd z,\quad E=\{(x,y,z)\mid x\ge 0,\ y\ge 0,\ z\ge 0,\ x+y+z\le 1\}
  \]
  を求めよ.
\par\noindent\dotfill\par
\noindent\textbf{【解答】}\par
まず $xy$ 平面への投影を考えると,
$0\le x\le 1,\ 0\le y\le 1-x$ である.
各 $(x,y)$ に対し $z$ は $0\le z\le 1-x-y$ なので,
体積はこの範囲で $1$ を積分すればよい
\begin{mathpad}
  \begin{align*}
    V
      &= \int_{0}^{1}\!\left(\int_{0}^{1-x}\!\left(\int_{0}^{1-x-y} 1\, \dd z\right)\dd y\right)\dd x \\
      &= \int_{0}^{1}\!\left(\int_{0}^{1-x} (1-x-y)\, \dd y\right)\dd x \\
      &= \int_{0}^{1}\!\left(\Biggl[\vphantom{\dfrac{\:y^{2}\:}{\:\:2\:}}(1-x)y-\dfrac{\:y^{2}\:}{\:\:2\:}\Biggr]{}_{\raisebox{-1.1ex}{$y=0$}}^{\raisebox{1.2ex}{$y=1-x$}}\right)\dd x \\
      &= \int_{0}^{1}\!\left(\dfrac{\:{(1-x)}^{2}\:}{\:\:2\:}\right)\dd x \\
      &= \dfrac{\:1\:}{\:\:6\:}
  \end{align*}
\end{mathpad}
\end{exproblem}

\begin{exproblem}{三重積分の具体計算}{ex:tetra-linear}
  \[
    I=\iiint_{E} (x+y+z)\, \dd x \dd y \dd z,\quad E=\{(x,y,z)\mid x\ge 0,\ y\ge 0,\ z\ge 0,\ x+y+z\le 1\}
  \]
  を求めよ.
\par\noindent\dotfill\par
\noindent\textbf{【解答】}\par
前問と同じ投影を用い, 同じ順序で計算する
先に $z$ で積分して一次式を整理し, その後 $y,x$ の順に計算する
\begin{mathpad}
  \begin{align*}
    I
      &= \int_{0}^{1}\!\left(\int_{0}^{1-x}\!\left(\int_{0}^{1-x-y} (x+y+z)\, \dd z\right)\dd y\right)\dd x \\
      &= \int_{0}^{1}\!\left(\int_{0}^{1-x}\!\left(\biggl[(x+y)z+\dfrac{\:z^{2}\:}{\:\:2\:}\biggr]{}_{\raisebox{-1.1ex}{$z=0$}}^{\raisebox{1.2ex}{$z=1-x-y$}}\right)\dd y\right)\dd x \\
      &= \int_{0}^{1}\!\left(\int_{0}^{1-x}\!\left((x+y)(1-x-y)+\dfrac{\:{(1-x-y)}^{2}\:}{\:\:2\:}\right)\dd y\right)\dd x \\
      &= \int_{0}^{1}\!\left(\int_{0}^{1-x}\!\left((1-x-y)-\dfrac{\:{(1-x-y)}^{2}\:}{\:\:2\:}\right)\dd y\right)\dd x \\
      &\quad \text{ここで }(x+y-1+1)(1-x-y)=1-x-y-{(1-x-y)}^{2} \text{ を用いて整理した} \\
      &= \int_{0}^{1}\!\left(\biggl[-\dfrac{\:{(1-x-y)}^{2}\:}{\:\:2\:}+\dfrac{\:{(1-x-y)}^{3}\:}{\:\:6\:}\biggr]{}_{\raisebox{-1.1ex}{$y=0$}}^{\raisebox{1.2ex}{$y=1-x$}}\right)\dd x \\
      &= \int_{0}^{1}\!\left(\dfrac{\:{(1-x)}^{2}\:}{\:\:2\:}-\dfrac{\:{(1-x)}^{3}\:}{\:\:6\:}\right)\dd x \\
      &= \biggl[-\dfrac{\:{(1-x)}^{3}\:}{\:\:6\:}+\dfrac{\:{(1-x)}^{4}\:}{\:\:24\:}\biggr]{}_{\raisebox{-1.1ex}{$x=0$}}^{\raisebox{1.2ex}{$x=1$}} \\
      &= \dfrac{\:1\:}{\:\:8\:}
  \end{align*}
\end{mathpad}
\end{exproblem}

\subsection{変数変換による重積分の計算}
\noindent\textbf{変数変換の基本公式}\par
2変数の変換
\[
  x=x(u,v),\quad y=y(u,v)
\]
が $1$ 対 $1$ 対応で滑らかに与えられるとき,
領域 $D$ に対する重積分は
\begin{mathpad}
  \[
    \iint_{D} f(x,y)\, \dd x \dd y
    = \iint_{D'} f\bigl(x(u,v),y(u,v)\bigr)\, \abs{\pdv{(x,y)}{(u,v)}}\, \dd u \dd v
  \]
\end{mathpad}
と変換できる. ここで $D'$ は $(u,v)$ 平面での対応領域であり,
ヤコビアンは
\[
  \pdv{(x,y)}{(u,v)}
  = \begingroup
    \setlength{\arraycolsep}{7pt}
    \renewcommand{\arraystretch}{1.2}
    \left|\begin{array}{cc}
      \textstyle \pdv{x}{u} & \textstyle \pdv{x}{v} \\
      \textstyle \pdv{y}{u} & \textstyle \pdv{y}{v}
    \end{array}\right|
    \endgroup
\]
である.

\begin{sidebox}{境界式の求め方}
  \begin{itemize}[leftmargin=2.2em]
    \item 領域の境界を方程式で表し, 変数変換を代入して $(u,v)$ の条件に直す
    \item 不等式の向きと範囲($\theta$ の区間など)を図で確認する
    \item 変換後の領域が単純になる順に, 条件を整理して積分範囲を決める
  \end{itemize}
\end{sidebox}

\begin{exproblem}{一次変換による領域の直方化}{ex:linear-change}
  変換
  \[
    x=2u+v,\quad y=u+2v
  \]
  \[
    I=\iint_{D} (x+y)\, \dd x \dd y,\quad D=\{(x,y)\mid x=2u+v,\ y=u+2v,\ 0\le u\le 1,\ 0\le v\le 1\}
  \]
  を求めよ.
  \par\noindent\dotfill\par
  \noindent\textbf{【解答】}\par
  変換は一次なので, 対応領域は $D'=\{(u,v)\mid 0\le u\le 1,\ 0\le v\le 1\}$ となる.
  まずヤコビアンを求めると,
  \[
    \pdv{(x,y)}{(u,v)}
    = \begin{vmatrix}
        2 & 1 \\
        1 & 2
      \end{vmatrix}
    = 3
  \]
  である. また $x+y=3u+3v$ なので
  \begin{mathpad}
    \begin{align*}
      I
        &= \int_{0}^{1}\!\left(\int_{0}^{1} (3u+3v)\cdot 3\, \dd v\right)\dd u \\
        &= 9\int_{0}^{1}\!\left(\left[uv+\dfrac{\:v^{2}\:}{\:\:2\:}\right]{}_{\raisebox{-1.1ex}{$v=0$}}^{\raisebox{1.2ex}{$v=1$}}\right)\dd u \\
        &= 9\int_{0}^{1}\!\left(u+\dfrac{\:1\:}{\:\:2\:}\right)\dd u \\
        &= 9\biggl[\dfrac{\:u^{2}\:}{\:\:2\:}+\dfrac{\:u\:}{\:\:2\:}\biggr]{}_{\raisebox{-1.1ex}{$u=0$}}^{\raisebox{1.2ex}{$u=1$}} \\
        &= 9
    \end{align*}
  \end{mathpad}
\end{exproblem}

\begin{exproblem}{極座標変換による円盤領域}{ex:polar-change}
  \[
    I=\iint_{D} (x^{2}+y^{2})\, \dd x \dd y,\quad D=\{(x,y)\mid x^{2}+y^{2}\le 1\}
  \]
  を求めよ.
  \par\noindent\dotfill\par
  \noindent\textbf{【解答】}\par
  極座標 $x=r\cos\theta,\ y=r\sin\theta$ を用いると,
  $x^{2}+y^{2}=r^{2}$, ヤコビアンは $r$ である.
  領域は $0\le r\le 1,\ 0\le \theta\le 2\pi$ となるので
  \begin{mathpad}
    \begin{align*}
      I
        &= \int_{0}^{2\pi}\!\left(\int_{0}^{1} r^{2}\cdot r\, \dd r\right)\dd \theta \\
        &= \int_{0}^{2\pi}\!\left(\biggl[\dfrac{\:r^{4}\:}{\:\:4\:}\biggr]{}_{\raisebox{-1.1ex}{$r=0$}}^{\raisebox{1.2ex}{$r=1$}}\right)\dd \theta \\
        &= \int_{0}^{2\pi}\!\left(\dfrac{\:1\:}{\:\:4\:}\right)\dd \theta \\
        &= \dfrac{\:\pi\:}{\:\:2\:}
    \end{align*}
  \end{mathpad}
\end{exproblem}

\begin{exproblem}{円の平行移動と変数変換の選択}{ex:shifted-circle}
  \[
    I=\iint_{D} (x^{2}+y^{2})\, \dd x \dd y,\quad D=\{(x,y)\mid x^{2}+y^{2}\le 2y\}
  \]
  を求めよ.
  \par\noindent\dotfill\par
  \noindent\textbf{【解答】}\par
  変数変換は, 領域の境界が簡単になるかを重視して選ぶ.
  本問では $x^{2}+y^{2}=2y$ は円 ${x}^{2}+{(y-1)}^{2}=1$ を表すので,
  極座標 $x=r\cos\theta,\ y=r\sin\theta$ を用いると,
  境界式 $x^{2}+y^{2}=2y$ は
  \[
    r^{2}=2r\sin\theta
  \]
  となる. ここで $r\ge 0$ より
  \[
    r=2\sin\theta
  \]
  が境界となる(図を参照). よって
  \[
    0\le \theta\le \pi,\quad 0\le r\le 2\sin\theta
  \]
  であり, ヤコビアンは $r$ である.
  \begin{figure}[H]
    \centering
    \includegraphics[width=0.52\linewidth]{img/fig1-4.pdf} %chktex 8
    \caption{境界 $x^{2}+y^{2}=2y$ と角度 $\theta$ の取り方}
    {}\label{fig:shifted-circle-boundary}
  \end{figure}
  \begin{mathpad}
    \begin{align*}
      I
      &= \int_{0}^{\pi}\!\left(\int_{0}^{2\sin\theta} r^{2}\cdot r\, \dd r\right)\dd \theta \\
      &= \int_{0}^{\pi}\!\left(\biggl[\dfrac{\:r^{4}\:}{\:\:4\:}\biggr]{}_{\raisebox{-1.1ex}{$r=0$}}^{\raisebox{1.2ex}{$r=2\sin\theta$}}\right)\dd \theta \\
      &= \int_{0}^{\pi} 4{\sin\theta}^{4}\, \dd \theta \\
      &= 4\cdot \dfrac{\:3\:}{\:4\:}\cdot \dfrac{\:\pi\:}{\:2\:} \\
      &= \dfrac{\:3\pi\:}{\:\:2\:}
    \end{align*}
  \end{mathpad}
  境界式が単純になる変数変換を選ぶと, 積分範囲の決定と計算が効率化される
  参考として, ウォリス積分の結果を偶数・奇数, $\sin/\cos$ ごとにまとめると
  \begin{mathpad}
    \[
      \setlength{\arraycolsep}{14pt}
      \renewcommand{\arraystretch}{1.45}
      \setlength{\extrarowheight}{2pt}
      \begin{array}{l|c|c}%chktex 44
        & \displaystyle \int_{0}^{\pi/2} {\sin{}^n\theta}\, \dd \theta & \displaystyle \int_{0}^{\pi/2} {\cos{}^{n}\theta}\, \dd \theta \\
        \hline%chktex 44
        n\:\text{が偶数} & \dfrac{\:n-1\:}{\:n\:}\cdot\dfrac{\:n-3\:}{\:n-2\:}\cdots\dfrac{\:1\:}{\:2\:}\cdot\dfrac{\:\pi\:}{\:2\:} & \dfrac{\:n-1\:}{\:n\:}\cdot\dfrac{\:n-3\:}{\:n-2\:}\cdots\dfrac{\:1\:}{\:2\:}\cdot\dfrac{\:\pi\:}{\:2\:} \\
        n\:\text{が奇数} & \dfrac{\:n-1\:}{\:n\:}\cdot\dfrac{\:n-3\:}{\:n-2\:}\cdots\dfrac{\:2\:}{\:3\:} & \dfrac{\:n-1\:}{\:n\:}\cdot\dfrac{\:n-3\:}{\:n-2\:}\cdots\dfrac{\:2\:}{\:3\:}
      \end{array}
    \]
  \end{mathpad}
  ここで二重階乗は
  \[
    (2m)!!=2m\cdot(2m-2)\cdots 2,\quad (2m-1)!!=(2m-1)\cdot(2m-3)\cdots 1
  \]
  を表す. また図形的な観点から, 
  \begin{figure}[H]
    \centering
    \includegraphics[width=0.66\linewidth]{img/fig1-6.pdf} %chktex 8
    \caption{対称性による区間分割}
    {}\label{fig:wallis-symmetry}
  \end{figure}
  \[\int_{0}^{\pi} {\sin{}^{n}\theta}\, \dd \theta = 2\int_{0}^{\pi/2} {\sin{}^{n}\theta}\, \dd \theta
  \] 
  である. 
\end{exproblem}

\begin{exproblem}{円柱座標による体積計算}{ex:cylindrical-change}
  \[
    V=\iiint_{E} 1\, \dd x \dd y \dd z,\quad E=\{(x,y,z)\mid x^{2}+y^{2}\le 1,\ 0\le z\le 2\}
  \]
  を求めよ.
  \par\noindent\dotfill\par
  \noindent\textbf{【解答】}\par
  円柱座標 $x=r\cos\theta,\ y=r\sin\theta,\ z=z$ を用いる.
  ヤコビアンは $r$ であり, 領域は
  $0\le r\le 1,\ 0\le \theta\le 2\pi,\ 0\le z\le 2$ となる.
  \begin{mathpad}
    \begin{align*}
      V
        &= \int_{0}^{2\pi}\!\left(\int_{0}^{1}\!\left(\int_{0}^{2} 1\cdot r\, \dd z\right)\dd r\right)\dd \theta \\
        &= \int_{0}^{2\pi}\!\left(\int_{0}^{1} 2r\, \dd r\right)\dd \theta \\
        &= \int_{0}^{2\pi}\!\left(\biggl[r^{2}\biggr]{}_{\raisebox{-1.1ex}{$r=0$}}^{\raisebox{1.2ex}{$r=1$}}\right)\dd \theta \\
        &= 2\pi
    \end{align*}
  \end{mathpad}
\end{exproblem}

\subsection{拡張された重積分とベータ関数}
これまでの重積分は有界領域で扱ったが, 半無限領域や無限領域でも同様に累次積分で計算できる.
その際は, 有界領域での計算に帰着させて極限を取ることが重要である.
さらに, ベータ関数により典型的な重積分が整理されるので, まとめて確認する.

\noindent\textbf{計算方法}\par
拡張された重積分は, まず領域を有界部分に切り取り, 極限を取ることで定義する.
収束が保証される場合に限り, 累次積分の順序交換が可能である.
被積分関数が不連続となる境界があるときは, その境界で領域を分割し, 各部分領域で計算して足し合わせる.

\noindent\textbf{ベータ関数の定義}\par
\begin{mathpad}
  \[
    B(p,q)=\int_{0}^{1} {(t)}^{p-1}{(1-t)}^{q-1}\, \dd t\quad (p>0,\ q>0)
  \]
\end{mathpad}
が成り立つ. 対称性 $B(p,q)=B(q,p)$ と, ガンマ関数との関係
\begin{mathpad}
  \[
    B(p,q)=\dfrac{\:\Gamma(p)\Gamma(q)\:}{\:\:\Gamma(p+q)\:}
  \]
\end{mathpad}
を用いると, 多くの重積分が整理できる.

\begin{exproblem}{半無限帯の拡張積分}{ex:improper-strip}
  \[
    I=\iint_{D} e^{-(x+y)}\, \dd x \dd y,\quad D=\{(x,y)\mid x\ge 0,\ 0\le y\le 1\}
  \]
  を求めよ.
\par\noindent\dotfill\par
\noindent\textbf{【解答】}\par
有界領域に切り取って
\[
  I(R)=\int_{0}^{R}\!\left(\int_{0}^{1} e^{-(x+y)}\, \dd y\right)\dd x
  \quad (R>0)
\]
とおく. まず $y$ で積分すると
\begin{mathpad}
  \[
    I(R)=\int_{0}^{R}\!\left(\biggl[-e^{-(x+y)}\biggr]{}_{\raisebox{-1.1ex}{$y=0$}}^{\raisebox{1.2ex}{$y=1$}}\right)\dd x
    =(1-e^{-1})\int_{0}^{R} e^{-x}\, \dd x
  \]
\end{mathpad}
となる. さらに
\begin{mathpad}
  \[
    I(R)=(1-e^{-1})\biggl[-e^{-x}\biggr]{}_{\raisebox{-1.1ex}{$x=0$}}^{\raisebox{1.2ex}{$x=R$}}
    =(1-e^{-1})(1-e^{-R})
  \]
\end{mathpad}
より, $R\to\infty$ の極限を取って
\[
  I=1-e^{-1}
\]
を得る. 被積分関数が非負であるため, 極限操作は単調収束により正当化される.
\end{exproblem}

\begin{exproblem}{ガウス積分の二重化}{ex:gauss-double}
  \[
    I=\int_{0}^{\infty} e^{-{(x)}^{2}}\, \dd x
  \]
  を求めよ.
\par\noindent\dotfill\par
\noindent\textbf{【解答】}\par
まず $I(a)=\int_{0}^{a} e^{-{(x)}^{2}}\, \dd x\ (a>0)$ とおき,
\[
  I(a){}^{2}=\int_{0}^{a}\!\left(\int_{0}^{a} e^{-\left({(x)}^{2}+{(y)}^{2}\right)}\, \dd y\right)\dd x
\]
とする. 第1象限で, 正方形 $0\le x\le a,\ 0\le y\le a$ は
半径 $a$ の円弧領域を含み, 半径 $\sqrt{2}\,a$ の円弧領域に含まれるので
\begin{mathpad}
  \[
    \iint_{x^{2}+y^{2}\le a^{2}} e^{-\left({(x)}^{2}+{(y)}^{2}\right)}\, \dd x \dd y
    \le I(a){}^{2}
    \le \iint_{x^{2}+y^{2}\le 2{(a)}^{2}} e^{-\left({(x)}^{2}+{(y)}^{2}\right)}\, \dd x \dd y
  \]
\end{mathpad}
が成り立つ(図\ref{fig:gauss-sandwich}).
極座標変換により
\begin{mathpad}
  \begin{align*}
    \int_{0}^{\pi/2}\!\left(\int_{0}^{a} e^{-{(r)}^{2}} r\, \dd r\right)\dd \theta
      &= \dfrac{\:\pi\:}{\:\:2\:}\biggl[-\dfrac{\:1\:}{\:\:2\:}e^{-{(r)}^{2}}\biggr]{}_{\raisebox{-1.1ex}{$r=0$}}^{\raisebox{1.2ex}{$r=a$}} \\
      &= \dfrac{\:\pi\:}{\:\:4\:}\left(1-e^{-{(a)}^{2}}\right)
  \end{align*}
\end{mathpad}
を得る. 同様に半径 $\sqrt{2}\,a$ の場合は
\[
  \dfrac{\:\pi\:}{\:\:4\:}\left(1-e^{-2{(a)}^{2}}\right)
\]
となるので, はさみうちの原理により
\[
  \lim_{a\to\infty} I(a){}^{2}=\dfrac{\:\pi\:}{\:\:4\:}
\]
が従う. よって
\[
  I=\dfrac{\:\sqrt{\pi}\:}{\:\:2\:}
\]
となる.
\begin{figure}[H]
  \centering
  \includegraphics[width=0.58\linewidth]{img/fig1-5.pdf} %chktex 8
  \caption{正方形領域と円領域によるはさみうち}
  {}\label{fig:gauss-sandwich}
\end{figure}
\end{exproblem}

\begin{exproblem}{ベータ関数型積分}{ex:beta-standard}
  次式を証明しなさい.
  \[
    J=\int_{0}^{1} {(t)}^{m}{(1-t)}^{n}\, \dd t
    =\dfrac{\:m!\,n!\:}{\:\:(m+n+1)!\:}\quad (m,n\ge 0)
  \]
\par\noindent\dotfill\par
\noindent\textbf{【解答】}\par
以下 $I(m,n)=\int_{0}^{1} {(t)}^{m}{(1-t)}^{n}\, \dd t$ とおく.\\
部分積分で $m$ を下げ, $I(0,n)$ に帰着させる. 途中で
\[
{(1-t)}^{n+1}={(1-t)}^{n}-t{(1-t)}^{n}
\] 
を用いて $I(m-1,n+1)$ を整理する.\\

部分積分の公式
\[
  \int f(x)g'(x)\, \dd x = f(x)g(x)-\int f'(x)g(x)\, \dd x
\]
を $f(x)={(t)}^{m}$, $g'(x)={(1-t)}^{n}$ に適用すると
\[
  I(m,n)=\biggl[-\dfrac{\:{(t)}^{m}{(1-t)}^{n+1}\:}{\:\:n+1\:}\biggr]{}_{\raisebox{-1.1ex}{$t=0$}}^{\raisebox{1.2ex}{$t=1$}}
  +\dfrac{\:m\:}{\:\:n+1\:}\int_{0}^{1} {(t)}^{m-1}{(1-t)}^{n+1}\, \dd t
\]
となる. 境界項は $t=0,1$ で消えるので
\[
  I(m,n)=\dfrac{\:m\:}{\:\:n+1\:}I(m-1,n+1)
\]
を得る. ここで
\[
  I(m-1,n+1)=\int_{0}^{1} {(t)}^{m-1}\bigl({(1-t)}^{n}-t{(1-t)}^{n}\bigr)\, \dd t
=I(m-1,n)-I(m,n)
\]
より
\[
  I(m,n)=\dfrac{\:m\:}{\:\:m+n+1\:}I(m-1,n)
\]
となる. これを $m$ 回繰り返すと
\[
  I(m,n)=\dfrac{\:m\:}{\:\:m+n+1\:}\times \dfrac{\:m-1\:}{\:\:m+n\:}\times\cdots\times\dfrac{\:1\:}{\:\:n+2\:}\times I(0,n)
\]
である. さらに
\[
  I(0,n)=\int_{0}^{1} {(1-t)}^{n}\, \dd t
  =\biggl[-\dfrac{\:{(1-t)}^{n+1}\:}{\:\:n+1\:}\biggr]{}_{\raisebox{-1.1ex}{$t=0$}}^{\raisebox{1.2ex}{$t=1$}}
  =\dfrac{\:1\:}{\:\:n+1\:}
\]
より
\[
  I(m,n)=\dfrac{\:m\:}{\:\:m+n+1\:}\times \dfrac{\:m-1\:}{\:\:m+n\:}\times\cdots\times\dfrac{\:1\:}{\:\:n+2\:}\times\dfrac{1}{n+1}\times \dfrac{n!}{n!}=\dfrac{\:m!\,n!\:}{\:\:(m+n+1)!\:}
\]
となる. 

したがって, 
\[
  J=I(m,n)=\dfrac{\:m!\,n!\:}{\:\:(m+n+1)!\:}
\]
\end{exproblem}

\begin{exproblem}{単体領域の典型計算}{ex:simplex-beta}
  \[
    K=\iint_{D} {(x)}^{m}{(y)}^{n}{(1-x-y)}^{l}\, \dd x \dd y,\quad D=\{(x,y)\mid x\ge 0,\ y\ge 0,\ x+y\le 1\}
  \]
  ただし $l,m,n$ は $0$ 以上の整数とする.
  \par\noindent\dotfill\par
\noindent\textbf{【解答】}\par
まず $y=(1-x)t$ とおいて $y$ 積分をベータ関数型に帰着し,
その後 $x$ の一変数積分を例題1.11の結果で評価する.
\\
領域条件より $0\le x\le 1$, $0\le y\le 1-x$ であるから
\[
  K=\int_{0}^{1}\!\left(\int_{0}^{1-x} {(x)}^{m}{(y)}^{n}{(1-x-y)}^{l}\, \dd y\right)\dd x
\]
と書ける. ここで $y=(1-x)t$ とおくと $\dd y=(1-x)\dd t$ であり,
置換後の対応は
\begin{mathpad}
  \[
    \begin{array}{c|l} %chktex 44
      y & 0\rightarrow 1-x \\
      \hline %chktex 44
      t & 0\rightarrow 1
    \end{array}
  \]
\end{mathpad}
である. よって
\begin{mathpad}
  \[
    \int_{0}^{1-x} {(y)}^{n}{(1-x-y)}^{l}\, \dd y
    =\int_{0}^{1} {(t)}^{n}{(1-x)}^{n}{\bigl((1-x)-(1-x)t\bigr)}^{l}(1-x)\, \dd t
  \]
\end{mathpad}
となり,
\begin{mathpad}
  \[
    \int_{0}^{1-x} {(y)}^{n}{(1-x-y)}^{l}\, \dd y
    ={(1-x)}^{n+l+1}\int_{0}^{1} {(t)}^{n}{(1-t)}^{l}\, \dd t
  \]
\end{mathpad}
を得る. 例題1.11より
\[
  \int_{0}^{1} {(t)}^{n}{(1-t)}^{l}\, \dd t=\dfrac{\:n!\,l!\:}{\:\:(n+l+1)!\:}
\]
なので
\[
  K=\dfrac{\:n!\,l!\:}{\:\:(n+l+1)!\:}\int_{0}^{1} {(x)}^{m}{(1-x)}^{n+l+1}\, \dd x
\]
となる. 再び例題1.11を用いると
\[
  \int_{0}^{1} {(x)}^{m}{(1-x)}^{n+l+1}\, \dd x
  =\dfrac{\:m!\,(n+l+1)!\:}{\:\:(m+n+l+2)!\:}
\]
であるから
\[
  K=\dfrac{\:m!\,n!\,l!\:}{\:\:(m+n+l+2)!\:}
\]
を得る.
したがって, 
\[
  K=\dfrac{\:m!\,n!\,l!\:}{\:\:(m+n+l+2)!\:}
\]
\end{exproblem}

\begin{sidebox}{ヤコビアンのまとめ}
  \begin{itemize}[leftmargin=2.2em]
    \setlength{\itemsep}{4pt}
    \setlength{\parsep}{2pt}
    \item 2変数一般変換: $\abs{\pdv{(x,y)}{(u,v)}}$
    \item 3変数一般変換: $\abs{\pdv{(x,y,z)}{(u,v,w)}}$
    \item $n$ 変数一般変換: $\abs{\pdv{(x_{1},\ldots,x_{n})}{(u_{1},\ldots,u_{n})}}$
    \item 極座標: $x=r\cos\theta,\ y=r\sin\theta,\ J=r$
    \item 円柱座標: $x=r\cos\theta,\ y=r\sin\theta,\ z=z,\ J=r$
    \item 球座標: $x=r\sin\phi\cos\theta,\ y=r\sin\phi\sin\theta,\ z=r\cos\phi,\ J=r^{2}\sin\phi$
  \end{itemize}
\end{sidebox}

\end{document}
