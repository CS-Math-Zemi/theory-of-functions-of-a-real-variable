
% ==============================================
% preamble.tex
% ユーザー定義のコマンド、環境、レイアウト設定などを記述
% ==============================================

% ---- レイアウト設定 (geometry) ----
\geometry{top=25mm,bottom=25mm,left=25mm,right=25mm}

% ---- 数式番号を章ごとに (1.1) のようにする ----
\numberwithin{equation}{section}

% ---- mathpad: 数式ブロックにだけ余白を付ける環境 ----
\newtcolorbox{mathpad}{%
  enhanced,
  frame hidden,
  boxrule=0pt,
  colback=white,
  opacityback=0,
  left=6mm,right=6mm,
  top=2mm,bottom=2mm,
  boxsep=0pt,
  before skip=2pt,
  after skip=2pt,
  before upper={\setlength{\jot}{5pt}},
}

% ---- pstep: (i) みたいな手順表示(改ページ可能版) ----
\newlength{\pstepindent}
\setlength{\pstepindent}{3.2em}

\ExplSyntaxOn %chktex 1
\NewDocumentCommand{\pstepformat}{m m}{%
  \str_case:nnF {#1}
    {
      {box}{\fbox{\textbf{#2}}}
      {bracket}{\textbf{[#2]}}
      {paren}{\textbf{(#2)}}
    }
    {\textbf{(#2)}}%
}
\ExplSyntaxOff%chktex 1
\NewDocumentEnvironment{pstep}{O{paren} m}{%
  \begin{list}{}{%
    \setlength{\leftmargin}{\pstepindent}%
    \setlength{\labelwidth}{\pstepindent}%
    \setlength{\labelsep}{0pt}%
    \setlength{\itemindent}{0pt}%
    \setlength{\topsep}{0pt}%
    \setlength{\partopsep}{0pt}%
    \setlength{\parsep}{0pt}%
    \setlength{\itemsep}{0pt}%
  }%
  \item[\pstepformat{#1}{#2}]%
  \leavevmode\par\nobreak\noindent\ignorespaces%
  % display数式まわりの詰め(必要なら調整)
  \setlength{\abovedisplayskip}{2pt}%
  \setlength{\belowdisplayskip}{2pt}%
  \setlength{\abovedisplayshortskip}{1pt}%
  \setlength{\belowdisplayshortskip}{1pt}%
}{%
  \end{list}%
}

% ---- \xoverline{...}: 少し長めの上線(複素共役などで使う) ----
\makeatletter
\newsavebox{\xoverline@boxA}
\newsavebox{\xoverline@boxB}
\newcommand*\xoverline[2][0.75]{%
  \sbox{\xoverline@boxA}{$\m@th#2$}%
  \setbox\xoverline@boxB\null%
  \ht\xoverline@boxB=\ht\xoverline@boxA%
  \dp\xoverline@boxB=\dp\xoverline@boxA%
  \wd\xoverline@boxB=#1\wd\xoverline@boxA%
  \sbox{\xoverline@boxB}{$\m@th\overline{\copy\xoverline@boxB}$}%
  \setlength\@tempdima{\wd\xoverline@boxA}%
  \addtolength\@tempdima{-\wd\xoverline@boxB}%
  \ifdim\wd\xoverline@boxB<\wd\xoverline@boxA%
    \rlap{\hskip 0.5\@tempdima\usebox{\xoverline@boxB}}%
    \usebox{\xoverline@boxA}%
  \else
    \usebox{\xoverline@boxB}%
  \fi
}
\makeatother

% ---- tcolorbox 互換: \tcbbreak が無い環境用 ----
\makeatletter
\providecommand{\tcbbreak}[1][]{%
  \par\pagebreak[2]\par
}
\makeatother

% ---- その他 演算子定義 ----
\DeclareMathOperator{\Arg}{Arg}

% ここに追加のコマンドを定義してください
\renewcommand{\dd}{\mathit{d}}
% --- preamble に追加 ---
\usepackage{titlesec}
%\usepackage[most]{tcolorbox}
\usepackage{tikz}
\usepackage{xcolor}
\tcbuselibrary{skins}
\usetikzlibrary{arrows.meta,shapes.callouts,decorations.pathmorphing}
% 色(適当に)
\definecolor{SecBlack}{HTML}{2B2B2B}
\definecolor{SubGray}{HTML}{E6E6E6}
\definecolor{RuleGray}{HTML}{9A9A9A}

% ---- emoji callout (math-safe, no spacing impact) ----
\tikzset{
  mathcallout bubble/.style={
    draw,
    ellipse callout,
    align=left,
    inner sep=3pt,
    font=\sffamily\bfseries\small,
  },
}
\newcommand{\mathcalloutemoji}[1]{\fontsize{18}{18}\selectfont\emotion{#1}}
\NewDocumentCommand{\mathcallout}{O{2.0em} O{2.1} O{0.15} O{10pt} O{1pt} m m}{%
  \smash{\rlap{\hspace{#1}\raisebox{0pt}[0pt][0pt]{%
    \tikz[baseline]{%
      \node (emo) at (0.0,0.6) {\mathcalloutemoji{#6}};%chktex 1
      \node[
        mathcallout bubble,
        callout absolute pointer={([xshift=#4,yshift=#5]emo.center)}
      ] at (#2,#3) {#7};
    }%
  }}}%
}

% ---- math underline annotation (wave/solid/double, color selectable) ----
\ExplSyntaxOn%chktex 1
\NewDocumentCommand{\mathunderline}{O{wave} O{red} m O{}}{%
  \str_case:nnF {#1}
    {
      {wave}{\def\mathunderlineDecor{\uwave}}
      {solid}{\def\mathunderlineDecor{\uline}}
      {double}{\def\mathunderlineDecor{\uuline}}
    }
    {\def\mathunderlineDecor{\uline}}%
  \def\mathunderlineBase{\mbox{$\displaystyle #3$}}%
  \def\mathunderlineBody{%
    \rlap{\textcolor{#2}{\mathunderlineDecor{\phantom{\mathunderlineBase}}}}%
    \mathunderlineBase%chktex 1
  }%
  \tl_if_blank:nTF {#4}
    {\mathunderlineBody}
    {\underset{\textcolor{#2}{\scriptsize #4}}{\mathunderlineBody}}%
}
\ExplSyntaxOff%chktex 1

% ---- nested underline annotation (bottom double underline + label) ----
\NewDocumentCommand{\mathnestunderline}{O{blue} m m}{%
  \underset{\textcolor{#1}{\scriptsize #2}}{%
    \begingroup
    \color{#1}\underline{\underline{\textcolor{black}{#3}}}%
    \endgroup
  }%
}

% ===== section(上のやつ:左に黒い角丸、右にタイトル+下に線) =====
\newcommand{\MySectionFormat}[1]{%
  \noindent
  % 左の黒い角丸("Chapter 9"の雰囲気)
  \tcbox[
    on line,
    arc=10pt,
    boxrule=0pt,
    colback=SecBlack,
    coltext=white,
    left=10pt,right=10pt,top=8pt,bottom=8pt
  ]{%
    \sffamily\bfseries
    \begin{tabular}{c}
      \raisebox{1.2ex}{Chapter}\\[-2pt]
      \raisebox{-1.0ex}{{\fontsize{32}{32}\selectfont \thesection}}
    \end{tabular}
  }%
\hspace{0.9em}%
\begin{minipage}[c]{\dimexpr\linewidth-4.2cm-0.9em\relax}
  {\bfseries\fontsize{28}{30}\selectfont #1}
\end{minipage}\par
  \vspace{0.3em}
  {\color{RuleGray}\rule{\linewidth}{0.6pt}}\par
}

\titleformat{\section}[block]
  {\normalfont}{}{0pt}{\MySectionFormat}

\titlespacing*{\section}{0pt}{2.0ex}{1.2ex}
% ===== subsection(見本寄せ:左だけ濃グレー帯+外枠角丸) =====
\newlength{\SubLeftW}
\setlength{\SubLeftW}{2.8cm} % 左帯の幅

\newcommand{\MySubsectionFormat}[1]{%
  \begin{tcolorbox}[
    enhanced,
    colback=white,
    colframe=RuleGray,
    boxrule=0.8pt,
    arc=14pt,
    boxsep=0pt,
    left=\dimexpr\SubLeftW+1.0em\relax,
    right=1.0em,
    top=0.55em,
    bottom=0.55em,
    valign=center,
    overlay={%
      \begin{tcbclipinterior}
        \fill[SecBlack]
          (frame.north west) rectangle ([xshift=\SubLeftW]frame.south west);
      \end{tcbclipinterior}
      \node[font=\bfseries\Large, text=white]
        at ([xshift=0.5\SubLeftW]frame.west) {\thesubsection};
    },
  ]%
    {\bfseries\Large #1}%
  \end{tcolorbox}%
}

\titleformat{\subsection}[block]{\normalfont}{}{0pt}{\MySubsectionFormat}
\titlespacing*{\subsection}{0pt}{1.2ex}{0.8ex}
