% ==================================================
% src/includes/protocol.tex
%  - Basic Packages
%  - Theorem-like environments (tcolorbox)
% ==================================================

% ----- 1. Basic Packages -----
\usepackage{luatexja}
\usepackage[hiragino-pron]{luatexja-preset}
\usepackage{amsmath,amssymb,amsfonts,mathtools}
\usepackage{amsthm}
\usepackage{bm}
\usepackage{physics}
\usepackage{siunitx}
\usepackage{graphicx}
\usepackage{float}
\usepackage{here}
\usepackage{geometry}
\usepackage{caption}
\usepackage{subcaption}
\usepackage{subfiles}
\usepackage{import}
\usepackage{xparse}
\usepackage{enumitem}
\usepackage{url}
\usepackage{cite}
\usepackage{emotion}
\usepackage[normalem]{ulem}
\usepackage{amsmath}
\usepackage[hidelinks]{hyperref}

% ----- 2. Fixes & Settings -----
\AtBeginDocument{\RenewCommandCopy\qty\SI}
\captionsetup[figure]{labelformat=simple,labelsep=colon,name={図}}
\captionsetup[table]{labelformat=simple,labelsep=colon,name={表}}

% ----- 3. TColorBox Definitions (User Design) -----
\usepackage{tcolorbox}
\tcbuselibrary{skins,breakable,theorems}

% =========================================
% tcolorbox 互換パッチ
% =========================================
\makeatletter
\pgfkeys{/tcb/.unknown/.code={%
  \edef\TcbUnknownName{\pgfkeyscurrentname}%
  \edef\TcbUnknownValue{\pgfkeyscurrentvalue}%
  \ifx\TcbUnknownValue\@empty%
    \tcbset{title={\TcbUnknownName}}%
  \fi
}}
\makeatother

% ===== Textbook boxes (tcolorbox) =====
% --- palette ---
\definecolor{TBteal}{HTML}{0EA5A4}
\definecolor{TBblue}{HTML}{2563EB}
\definecolor{TBgreen}{HTML}{16A34A}
\definecolor{TBorange}{HTML}{F97316}
\definecolor{TBred}{HTML}{DC2626}
\definecolor{TBpurple}{HTML}{7C3AED}
\definecolor{TBgray}{HTML}{374151}
\definecolor{TByellow}{HTML}{FDE047}
\definecolor{TBbrown}{HTML}{D97706}
\definecolor{TBcyan}{HTML}{04F6F6}

% --- base styles ---
\tcbset{
  tbBase/.style n args={1}{%
    enhanced,
    boxrule=0.8pt,
    arc=1.2mm,
    left=1.4mm,right=1.4mm,top=1.1mm,bottom=1.1mm,
    before skip=8pt, after skip=8pt,
    colback=white,
    fonttitle=\bfseries,
    title={#1},
  },
  tbBaseBreak/.style n args={1}{%
    tbBase={#1},
    breakable,
  },
  tbFrame/.style n args={2}{%
    tbBaseBreak={#2},% ← 外側は breakable
    frame hidden,
    colback=#1!3,
    borderline west={2.2pt}{0pt}{#1!80!black},
    title filled,
    colbacktitle=#1!12,
    coltitle=#1!80!black,
  },
  tbSide/.style n args={2}{%
    tbBase={#2},% ← 内側は breakable にしない
    frame hidden,
    colback=#1!3,
    borderline west={2.2pt}{0pt}{#1!80!black},
    title filled,
    colbacktitle=#1!12,
    coltitle=#1!80!black,
  },
  tbSideBreak/.style n args={2}{%
    tbBaseBreak={#2},% ← breakable
    frame hidden,
    colback=#1!3,
    borderline west={2.2pt}{0pt}{#1!80!black},
    title filled,
    colbacktitle=#1!12,
    coltitle=#1!80!black,
  },
}

% --- Guard for multiple loading ---
\makeatletter
\@ifundefined{protocol@loaded}{%
  \def\protocol@loaded{}%
}{%
  \endinput
}%
\makeatother

% --- Environments ---
\NewDocumentEnvironment{sidebox}{ O{TBgray} m }
  {\begin{tcolorbox}[tbSideBreak={#1}{#2}]}
  {\end{tcolorbox}}

\newtcbtheorem[number within=section]{theorem}{定理}
  {tbFrame={TBblue}{定理}}{th}

\newtcbtheorem[number within=section]{definition}{定義}
  {tbFrame={TBgreen}{定義}}{def}

\newtcbtheorem[number within=section]{proposition}{命題}
  {tbFrame={TBgreen}{命題}}{prop}

\newtcbtheorem[number within=section]{example}{例}
  {tbFrame={TBorange}{例}}{ex}

\newtcbtheorem[number within=section]{remark}{注意}
  {tbSide={TBred}{注意}}{rm}

\newtcbtheorem[number within=section]{exercise}{演習}
  {tbFrame={TBpurple}{演習}}{qs}

\newtcbtheorem[number within=section]{exproblem}{例題}
  {tbFrame={TBteal}{例題}}{exprob}

\newtcbtheorem[number within=section]{character}{性質}
  {tbFrame={TByellow}{性質}}{cha}

\newtcbtheorem[no counter]{solution}{解答}
  {tbSide={TBgray}{解答}}{sol}

\newtcbtheorem[no counter]{tproof}{証明}
  {tbFrame={TBbrown}{証明}}{prf}

% 互換用 sbox
\makeatletter
\@ifundefined{sbox}{%
  \NewDocumentEnvironment{sbox}{ O{TBgray} m }
    {\begin{tcolorbox}[tbSide={#1}{#2}]}
    {\end{tcolorbox}}%
}{}%
\makeatother
